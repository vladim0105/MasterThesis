\chapter{Conclusion \& Future Work}
\label{sec:conclusion}
\section{Summary}
Smart surveillance systems have seen increased demand in the past few years. Modern smart surveillance systems depend on deep learning for their intelligence. However, it has been shown that deep learning faces several challenges. Some challenges are explainability, noise-tolerance, training data volume, and concept drift. While there are works that attempt to address these challenges, it is still important to look elsewhere for learning algorithms which do not face those same issues. One of them is HTM theory which introduces a learning algorithm that estimates the learning mechanism in the neocortex.
\par
Unlike deep learning, HTM works by using SDR to represent data and learns through Hebbian like learning. This gives it the property of noise-tolerance and on-line learning, meaning it can handle concept drift. A natural question to ask is whether it can be used for anomaly detection in videos. It has been shown that HTM performs well in low-dimensional data such as temperature data. However, it has also been shown that it performs poorly in high dimensional data such as images and videos due to the difficulty in converting that type of data to SDRs.
\par
With that in mind, this thesis attempts to make it possible to conduct anomaly detection in videos with the introduction of Grid HTM. Instead of having a single HTM model run anomaly detection on videos, Grid HTM divides the frames into a grid where each individual cell has its own HTM model. This makes the entire system more invariant, gives it increased flexibility, and increases explainability. The videos themselves can be converted into SDRs using techniques such as deep learning segmentation and binary thresholding.
\par
This thesis then introduces three experiments, each aiming to prove different aspects of HTM and Grid HTM. The first experiment is a simple video of a bouncing ball, which has the aim of proving that a single HTM model can perform anomaly detection on a simple and controlled video, it then does the same but with Grid HTM instead to prove that it still works. The second experiment aims to showcase the capabilities of Grid HTM on a complex surveillance video. It shows that, given the limited and noisy data, Grid HTM is able to learn the norm and backs it up with concrete examples. It also showed that Grid HTM was able to detect video segments. The third experiment explores the capability of Grid HTM for detecting segments using a very noisy sperm video dataset.
\par
The experiments show that with proper data and further refinements, Grid HTM and other HTM based systems could be powerful anomaly detection systems.
\section{Contributions?}
\section{Future Work}
As mentioned in \autoref{sec:introduction}, one of the main limitations is the lack of video datasets suited for anomaly detection by HTM. Therefore, the most important future work would be to create such a dataset. The videos would optimally be several days long and contain anomalies such as car accidents, jaywalking, and other similar anomalous behaviors.
\par
For Grid HTM, more time should be spent exploring other aggregation functions so that the aggregated anomaly score can be used more efficiently.  Additionally, it would be a big benefit to create an algorithm which can decide the parameters for each cell during the calibration phase. Finally, experiments should be performed to validate the possibility of having the TM in each cell grow synapses to neighboring cells in order to solve the issue with unstable anomaly output, which was mentioned in \autoref{sec:stabilizing_anomaly_output}.
\par
Another important field to research is a tighter integration between HTM and deep learning. This way it could be possible to leverage the self-supervision and noise resilience property of HTM, together with the powerful feature extraction and representation of deep learning approaches. Effectively combining the best of both approaches while eliminating the disadvantages that have been mentioned in \autoref{sec:background}.
\par
HTM theory is in constant development, especially as the understanding of the brain grows. Future work would therefore include keeping up to date on the latest developments within HTM theory, and update the model and add new systems accordingly.