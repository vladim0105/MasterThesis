\chapter{Conclusion \& Future Work}
\label{sec:conclusion}
\section{Summary}
Smart surveillance systems have seen increased demand in the past few years. Modern smart surveillance systems depend on deep learning for their intelligence. However, it has been shown that deep learning faces several challenges. Some challenges are explainability, noise-tolerance, training data volume, and concept drift. While there are works that attempt to address these challenges, it is still important to look elsewhere for learning algorithms which do not face those same issues. One of them is HTM theory which introduces a learning algorithm that estimates the learning mechanism in the neocortex.
\par
Unlike deep learning, HTM works by using SDR to represent data and learns through Hebbian like learning. This gives it the property of noise-tolerance and online learning, meaning it can handle concept drift. A natural question to ask is whether it can be used for anomaly detection in videos. It has been shown that HTM performs well in low-dimensional data such as temperature data. However, it has also been shown that it performs poorly in high dimensional data such as images and videos due to the difficulty in converting that type of data to SDRs.
\par
With that in mind, this thesis attempts to make it possible to conduct anomaly detection in videos with the introduction of Grid HTM. Instead of having a single HTM model run anomaly detection on videos, Grid HTM divides the frames into a grid where each individual cell has its own HTM model. This makes the entire system more invariant, gives it increased flexibility, and increases explainability. The videos themselves can be converted into SDRs using techniques such as deep learning segmentation and binary thresholding.
\par
This thesis then introduces three experiments, each aiming to prove different aspects of HTM and Grid HTM. The first experiment is a simple video of a bouncing ball, which has the aim of proving that a single HTM model can perform anomaly detection on a simple and controlled video, it then does the same but with Grid HTM instead to prove that it still works. The second experiment aims to showcase the capabilities of Grid HTM on a complex surveillance video. It shows that, given the limited and noisy data, Grid HTM is able to learn the norm and backs it up with concrete examples. It also showed that Grid HTM was able to detect video segments. The third experiment explores the capability of Grid HTM for detecting segments using a very noisy sperm video dataset.
\section{Contributions}
As introduced in \autoref{sec:introduction}, this thesis achieved three objectives that would help answer the thesis question. The objectives and how they were achieved are as follows:
\par
\paragraph*{Objective 1} \emph{Introduce HTM and give a deep understanding of the inner workings, the strengths, and the weaknesses. While also being friendly to readers with a machine learning background.}
\par
This objective was achieved in \autoref{sec:background}, where HTM was explained. It was explained in a straight-forward manner with references to detailed figures. It also covered common misconceptions and included information that is hard to come by. The chapter also highlighted the importance of the HTM community by including information directly from community discussions.
\paragraph*{Objective 2} \emph{Develop and outline a theoretically sound architecture so that HTM can be applied for anomaly detection in complex videos.}
\par
This objective was achieved in \autoref{sec:grid_htm}, where Grid HTM was introduced. It was introduced with the purpose of performing anomaly detection in videos. The chapter made sure that Grid HTM followed the rules that an HTM model should follow. It also introduced challenges and issues, and followed up with remedies.
\paragraph*{Objective 3} \emph{Perform experiments, discuss the results, and lay out potential future work. The experiments will vary in difficulty, complexity, and will focus on different use cases.}
\par
This objective was achieved in \autoref{sec:experiments}, where three different experiments were performed. The first experiment showcased that HTM and Grid HTM can indeed perform on simple and clean videos. The second experiment showcased the performance of Grid HTM on a complex surveillance video, which showed promising results. The third experiment showcased the ability of Grid HTM to detect segments in a video, for increased challenge noisy videos of sperm were used.
\par
Now that the three objectives have been achieved, it is possible to answer the thesis question: \textbf{Is HTM viable for anomaly detection in videos?} The experiments show that with proper data and further refinements, Grid HTM and other HTM based architectures could be used for anomaly detection systems for videos.
\section{Ethics}
This thesis introduces an architecture for performing surveillance on potentially a massive scale, and could even be tied to some sort of social credit system. This raises ethical questions regarding mass surveillance that highlight how the work done in this thesis can be misused. Therefore, even if the work done in this thesis is nowhere near perfect and far from ready to be used in production, one should still keep the ethical aspects in mind when pursuing this type of research.
\par
It should also be reiterated that HTM is the result of research from a private company, which is why one should be extra critical of their work. Another aspect is that unlike deep learning, which is democratized, HTM is still very centralized. This means that Numenta has a lot of influence within the development of HTM, which could be misused for the purpose of promoting the goals of the company.
\section{Future Work}
Seeing as this thesis presents a novel approach, there is naturally a lot of future work that can be done.
\subsection*{Dataset}
As mentioned in \autoref{sec:introduction}, one of the main limitations is the lack of video datasets suited for anomaly detection by HTM. Therefore, the most important future work would be to create such a dataset. The videos would optimally be several days long and contain anomalies such as car accidents, jaywalking, and other similar anomalous behaviors.
\par
\subsection*{Grid HTM}
For Grid HTM, more time should be spent exploring other aggregation functions so that the aggregated anomaly score can be used more efficiently. One could use deep learning for this purpose or perhaps use another layer of HTM, the possibilities are endless.
\par
Additionally, it would be a big benefit to create an algorithm which can decide the parameters for each cell during the calibration phase. It is also possible to improve explainability and robustness by implementing a measure of certainty for each cell.
\par
Depth vision or 3D vision should be experimented with, as the depth information could be valuable for anomaly detection in surveillance. With voxels, this could be used similarly to 2D segmentation, where there could be an extra SDR for each layer of depth in the voxelized 3D image.
\par
Finally, experiments should be performed to validate the possibility of having the TM in each cell grow synapses to neighboring cells in order to solve the issue with unstable anomaly output, which was mentioned in \autoref{sec:stabilizing_anomaly_output}.
\subsection*{HTM and Deep Learning}
Another important field to research is a tighter integration between HTM and deep learning. This way it could be possible to leverage the self-supervision and noise resilience property of HTM, together with the powerful feature extraction and representation of deep learning approaches. Effectively combining the best of both approaches while eliminating the disadvantages that have been mentioned in \autoref{sec:background}.
\par
\subsection*{Research Updates}
HTM theory is in constant development, especially as the understanding of the brain grows. Future work would therefore include keeping up to date on the latest developments within HTM theory and neuroscientifical research, and update the model and add new systems accordingly.