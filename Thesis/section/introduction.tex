\chapter{Introduction}
As the global need for security and automation increases, many entities seek to use video anomaly detection systems. Their uses can vary from detecting faulty products on an assembly line or detecting car incidents on the highway, and everything in between. Leveraging modern computer vision, modern anomaly detection systems play an important role in increasing monitoring efficiency and reducing the need for expensive live monitoring.\par
Despite the major progress within the field of Machine Learning, there are still many tasks where humans outperform algorithms. The reason for this probably lies in the difference between how humans and machine learning algorithms represent data and learn. Most machine learning algorithms use a dense representation of data and apply back-propagation in order to learn. Human learning happens in the neocortex, where the most popular theory is that the neocortex uses a sparse representation and performs Hebbian-style learning. For the latter, there is a growing field of machine learning dedicated to replicating the inner mechanics of the neocortex, namely Hierarchical Temporal Memory (HTM) theory. \par
\textbf{This thesis proposes an architecture which combines both deep learning image segmentation and HTM in order to perform anomaly detection on surveillance videos. }
\section{Limitations}
This thesis is novel and is therefore limited on several fronts. One of the main limitations is the lack of labeled anomaly data that fits the problem statement. Another problem is the lack of works related to applying HTM on video-based problems.
\section{Contributions}
This thesis contributes in multiple ways. Not only does it present a novel way to apply HTM to video-based problems, it also uncovers the reasoning behind the design decision that were made as well as providing thorough analysis. This thesis also acts as an organization of HTM related research and ties it into deep learning.
Github contribution.
